\maketitle


\section{Conslusion} 
%在本文中,我们介绍了FactExplorer, 一个用来辅助用户高效便捷地探索事件空间的系统。本系统,事件被自动从表格数据中提取。一个用来双因素(可视化样式和逻辑性)事件嵌入的方法被引入来将事件嵌入到事件空间,事件空间提供了所有事件地概览和每个事件的上下文。一个多故事线生成算法被设计来生成故事干线和支线。事件空间中的事件被合理地组织以便捷用户的探索和增强用户对事件空间的记忆。一些交互组件在系统中也被实现来支持用户灵活地编辑事件和事件故事线。通过两个个实例案例和一个用户研究,对该技术进行了评估。评价显示了FactExplorer的强大能力,并揭示了当前系统的几个局限性,这将在未来加以解决。
In this paper, we introduce FactExplorer, a system designed to help users efficiently and conveniently explore and analyze fact space. 
In this system, entire facts are automatically extracted from tabular data. 
A two-factor (visual style and logic) fact embedding approach is introduced to embed facts into the fact space, which provides an overview of all facts and the context of each fact. 
A multi-storyline generation algorithm is also designed to generate multiple perspectives storylines. The whole facts are organized well to promote exploration and deepen the impression of the fact space on the user. 
Certain interactive components are also implemented to support users to flexibly edit facts and storylines. 
These techniques proposed in this work is evaluated through two case studies and a user study. Finally, we discuss several limitations of the current system, which will be addressed in the future.